\documentclass[10pt,letterpaper]{article}
\usepackage[utf8]{inputenc}
\usepackage{amsmath}
\usepackage{amsfonts}
\usepackage{amssymb}
\usepackage{graphicx}
\author{Abraham Toriz Cruz}
\title{Implementation of Map-Matching}
\begin{document}

\maketitle

\begin{abstract}
TODO write an abstract
\end{abstract}

\begin{section}{Algorithm}

From the article.

\begin{enumerate}
	\item For each GPS position $z_i$, a set of candidate roads $\{r_i^0, r_i^1, \dots, r_i^n\}$ is determined, i.e. a set of all roads within a certain search-radius, for example 200m, around $z_i$.

	\item For each candidate road $r_i^j$, a \emph{measurement probability} is calculated which reflects how likely it is for a vehicle on road $r_i^j$ to emit a GPS measurement having position $z_i$. The probability depends on GPS error characteristics and decreases with increasing distance between road and GPS position.

	\item For each candidate road $r_i^j$ of a GPS position $z_i$, the \emph{transition probability} to all candidate roads $\{r_{i+1}^0, r_{i+1}^1, \dots, r_{i+1}^m\}$ of the next GPS position $z_{i+1}$ is calculated. The transition probability is an exponential function of the difference between the route length and the great circle distance between $z_i$ and $z_{i+1}$. The transition probability calculation therefore requires a shortest path routing between each pair of candidate roads, which is a computationally expensive operation.

	\item The Viterbi dynamic programming algorithm is applied to determine the map-matching solution by selecting the sequence of candidate roads which yields the overall best explanation for the observed GPS trajectory.
\end{enumerate}

\end{section}

\begin{section}{Important things to consider}
\begin{enumerate}
	\item Currently my approach is to use OSM street data, so the search graph is built by the nodes and not by lines. The search of roads within a radius is performed accordingly.
	\item The search radius will say how many routes per GPS position you have, thus changing the search graph size.
	\item For the test route a radius of $100m$ gives a minimum of one graph node and a maximum of 97, $150m$ give a minimum of 3 and maximum of 149, and $200m$ a minimum of 16 and a maximum of 236 nodes.
	\item You want the solution node to be within the search radius, but a large radius will make the search graph too big.
\end{enumerate}
\end{section}

\begin{section}{Tecnology}

\begin{enumerate}
	\item the Python programming language.

	\item Redis database.
\end{enumerate}

\end{section}

\end{document}
